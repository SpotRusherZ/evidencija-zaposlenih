\documentclass[a4paper]{article}

\usepackage{color}
\usepackage{url}
\usepackage[T2A]{fontenc} 
\usepackage[utf8]{inputenc}
\usepackage{graphicx}

\usepackage[english,serbian]{babel}
\usepackage[unicode]{hyperref}
\hypersetup{colorlinks,citecolor=red,filecolor=green,linkcolor=blue,urlcolor=blue}

\title{IS_Firme}
\date{November 2023}

\begin{document}

\begin{titlepage}
    \centering

    \newcommand{\HRule}{\rule{\linewidth}{0.5mm}}
    \center
    \textup{\Large Univerzitet u Beogradu\\Matematički fakultet}\\[1.5cm]
    \textup{\Large Projekat iz predmeta Informacioni sistemi 2022/2023.}\\[0.4cm]

    \HRule \\[0.4cm]
    { \huge \bfseries Informacioni sistem firme}\\[0.4cm]
    \HRule \\[1.1cm]
\end{titlepage}

\newpage
\renewcommand*\contentsname{Sadržaj:}
\tableofcontents

\newpage
\section{Analiza sistema}
\subsection{Uvod i osnovna ideja}
Ideja projekta je pravljenje informacionog sistema koji bi se koristio za vođenje evidencije o zaposlenima u jednoj programerskoj firmi. Ono što to podrazumeva je beleženje radnog vremena, beleženje odmora i slobodnih dana, rada od kuće, kao i informacije o benefitima čije troškove pokriva firma.

\subsection{Korisnici sistema}
Korisnici sistema su:
\begin{enumerate}
    \item \textbf{Zaposleni} \newline
    Zaposleni su najbrojniji korisnici sistema. Svakom zaposlenom je dodeljen tim kome on pripada, zaposleni može da menja timove. Funkcionalnosti koje će sistem pružiti radnicima je prijavljivanje ostvarenog radnog vremena, prijavljivanje odmora, podnošenje zahteva za rad od kuće i podnošenje zahteva za korišćenje benefita.
    \item \textbf{Vođa tima} \newline
    Vođa tima upravlja timom koji mu je dodeljen, on je i sam zaposleni. U svakom trenutku mu se njegovom timu može dodeliti novi član. Funkcionalnosti koje sistem pruža vođi tima su odobravanje zahteva zaposlenog, kreiranje zahteva za re-evaluaciju zaposlenog, zahtev za korišćenje benefita za nekog zaposlenog.
    \item \textbf{Menadžer ljudskih resursa} \newline 
    Menadžer ljudskih resursa je zadužen za evidentiranje informacija koje su bitne za sve zaposlene. Funkcionalnosti koje sistem pruža menadžeru za ljudske resurse su obračun plata, unos važnih datuma i dodela odobrenih benefita i bonusa.
    \item \textbf{Administrator} \newline
    Administrator je zadužen za odrzavanje sistema. Njegove dužnosti tokom korišćenja sistema su kreiranje naloga zaposlenima, brisanje naloga ljudima koji su prekinuli radni odnos, omogućavanje ponovnog pristupa nalogu pri zahtevu zaposlenog i obezbeđivanje opreme za rad. 
\end{enumerate}

\subsection{Udruženja}
Udruženja predstavljaju entitet unutar informacionog sistema, kome se pridružuju zaposleni koji dele zajednički interes. \newline
Informacioni sistem obezbedjuje zasebnu stranicu koja pripada udruženju, koja se može koristiti povodom:
\begin{enumerate}
    \item Diskusija u okviru domena interesa udruženja.
    \item Deljenja informacija i znanja među članovima.
    \item Organizovanje događaja.
\end{enumerate}

Za osobu zaposlenu u kompaniji, bez obzira na ulogu u okviru iste, važe sledeća pravila u okvriu informacionog sistema:
\begin{itemize}
    \item Dozvoljeno je kreiranje udruženja, uz odgovarajuće odobrenje od strane menadžera ljudskih resursa.
    \item Dozvoljeno je članstvo u proizvoljnom broju udruženja.
    \item Dozvoljeno je kreiranje događaja, uz odgovarajuće odobrenje od strane kreatora (vlasnika) udruženja.
\end{itemize}

\end{document}
