\documentclass[a4paper]{article}

\usepackage{color}
\usepackage{url}
\usepackage[T2A]{fontenc} 
\usepackage[utf8]{inputenc}
\usepackage{graphicx}

\usepackage[english,serbian]{babel}

\begin{document}

\section{Slučaj upotrebe: Unošenje radnih sati}
\begin{enumerate}
    \item \textbf{Kratak opis:} Zaposleni unosi broj odrađenih ranih sati za konkretan dan kao i opis odrađenog posla.
    \item \textbf{Učesnici:}
        \begin{itemize}
            \item Zaposleni
        \end{itemize}
    \item \textbf{Preduslovi:} Zaposleni je registrovani korisnik sistema i nije unet odrađen broj radnih sati za dan koji korisnik popunjava.
    \item \textbf{Postuslovi:} Odrađen broj radnih sati za konkretan dan je sačuvan u sistemu. Baza je ažurirana. Na stranici se prikazuje novi tiket koji je u nerazrešenom statusu.
    \item \textbf{Osnovni tok:}
        \begin{enumerate}
            \item Zaposleni otvara stranicu gde se prikazuju dani i uneti radni sati za te dane.
            \item Pritiska dugme za dodavanje novog dana i broja unetih radnih sati.
            \item Prikazuje se forma koji sadrži polja za odabir dana, broj odrađenih radnih sati, i opis odrađenog posla.
            \item Zaposleni popunjava formu.
            \item Potvrđuje unos za taj dan.
            \item Sistem vrši obradu podataka.
            \item Sistem čuva unete podatke u bazi podataka.
            \item Na stranici se dodaje tiket koji je u nerazrešenom statusu i informacije koje je korisnik uneo za taj dan.
        \end{enumerate}
    \item \textbf{Alternativni tokovi:}
        \begin{enumerate}
            \item \textbf{Zaposleni je odabrao nevalidan dan}. Ukoliko u koraku (d) korisnik odabere dan za koji je uneo radne sate ili dan koji je ispred trenutnog dana, prikazuje mu se odgovarajuća poruka o grešci. Proces se nastavlja u koraku (d).
            \item \textbf{Zaposleni je odabrao nevalidan broj radnih sati}. Ukoliko u koraku (d) zaposleni unese radne sate koji predstavljaju negativan broj, prikazuje mu se odgovarajuća poruka o grešci. Proces se nastavlja u koraku (d).
        \end{enumerate}
    \item \textbf{Podtokovi:} /
    \item \textbf{Specijalni zahtevi:} /
    \item \textbf{Dodatne informacije:} Polje za unosenje broja radnih sati je obavezno.
\end{enumerate}

\includegraphics[scale=0.5]{PrijavljivanjeRadnogVremena.jpg}

\section{Slučaj upotrebe: Izmena tiketa}
\begin{enumerate}
    \item \textbf{Kratak opis:} Zaposleni menja informacije u tiketu za konkretan dan.
    \item \textbf{Učesnici:}
        \begin{itemize}
            \item Zaposleni
        \end{itemize}
    \item \textbf{Preduslovi:} Tiket za dan čije informacije korisnik menja je prikazan na stranici, postoji u sistemu i u nerazrešenom je statusu.
    \item \textbf{Postuslovi:} Informacije su ažurirane u sistemu i korisnik vidi promene na tiketu za dan koji je ažurirao.
    \item \textbf{Osnovni tok:}
        \begin{enumerate}
            \item Zaposleni otvara stranicu gde se prikazuju dani i uneti radni sati za te dane.
            \item Pritiska odgovarajuće dugme za izmenu podataka na tiketu.
            \item Na tiketu polja za unošenje radnih sati i opis odrađenog posla postaju moguća za izmenu.
            \item Zaposleni menja odgovarajuća polja.
            \item Potvrđuje izmene.
            \item Sistem vrši obradu podataka.
            \item Sistem čuva unete podatke u bazi podataka.
            \item Na stranici se prikazuje tiket u izmenjenom obliku.
        \end{enumerate}
    \item \textbf{Alternativni tokovi:}
        \begin{enumerate}
            \item \textbf{Zaposleni je krenuo da odabira nedozvoljen broj dana za odmor}. Ukoliko u koraku (d) zaposleni unese radne sate koji predstavljaju negativan broj, prikazuje mu se odgovarajuća poruka o grešci. Proces se nastavlja u koraku (d).
        \end{enumerate}
    \item \textbf{Podtokovi:} /
    \item \textbf{Specijalni zahtevi:} /
    \item \textbf{Dodatne informacije:} Polje za unosenje broja radnih sati je obavezno.
\end{enumerate}
\includegraphics[scale=0.5]{IzmenaTiketa.jpg}

\section{Slučaj upotrebe: Prijavljivanje neradnih dana}
\begin{enumerate}
    \item \textbf{Kratak opis:} Zaposleni unosi u sistem neradne dane sto mogu biti dani odmora ili odsustvo.
    \item \textbf{Učesnici:}
        \begin{itemize}
            \item Zaposleni
        \end{itemize}
    \item \textbf{Preduslovi:} Zaposleni nije iskoristio ukupan broj dana od odmora.
    \item \textbf{Postuslovi:} Informacije su ažurirane u sistemu i korisnik vidi koliko mu je dana od odmora ostalo.
    \item \textbf{Osnovni tok:}
        \begin{enumerate}
            \item Zaposleni otvara stranicu za slobodne dane, gde mu se prikazuje ukupan broj slobodnih dana, kao i broj dana koji mu je ostao.
            \item Pritiskom na dugme za unosenje slobodnih dana, otvara mu se forma.
            \item Na formi bira jednu od dve sledece opcije: odmor, plaćeno odsustvo.
            \item Ako je uneta opcija za dane odmora, izvršava se podtok P1, inače se izvršava podtok P2.
            \item Nakon popunjavanja forme, pritiska dugme za potvrdu unosa podataka.
            \item Na stranici se prikazuje azuriran broj dana za odmor i plaćeno odsustvo.
        \end{enumerate}
    \item \textbf{Podtokovi:}
        \begin{enumerate}
            \item Dani odmora
            \begin{enumerate}
                \item Prikazuje se forma koja sadrži polje za odabir dana za odmor.
                \item Klikom na polje za odabir dana, zaposlenom se prikazuju unapred određeni dani za odmor.
                \item Zaposleni selektuje dane.
                \item Potvrdnim klikom, prikazuje mu se forma za opcioni unos poruke za zaposlene koji žele da ga kontaktiraju. 
            \end{enumerate}
            \item Plaćeno odsustvo
            \begin{enumerate}
                \item Prikazuje se forma koja sadrži polje za odabir dana za plaćeno odsustvo kao i polje gde se navodi razlog.
            \end{enumerate}  
    \end{enumerate}
    \item \textbf{Alternativni tokovi:}
        \begin{enumerate}
            \item \textbf{Zaposleni je iskoristio dane plaćenog odsustva}. Ukoliko u koraku iii. podtoku (a) zaposleni selektuje dan kojim bi premašio dozvoljen broj dana za odmor, izašla bi mu poruka o grešci na ekranu, i u istom koraku se nastavlja proces.
            \item \textbf{Zaposleni je krenuo da odabira nedozvoljen broj dana za odmor}. Ukoliko u koraku (d) osnovnog toka zaposleni odabere opciju korišćenja plaćenog odsutva a pritom je te dane iskoristio, izašla bi mu poruka o grešci na ekranu, i u istom koraku se nastavlja proces.
        \end{enumerate}
    \item \textbf{Specijalni zahtevi:} /
    \item \textbf{Dodatne informacije:} /.
\end{enumerate}


\end{document}
