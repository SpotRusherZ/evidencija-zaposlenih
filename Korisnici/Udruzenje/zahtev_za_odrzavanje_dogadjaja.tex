\documentclass[a4paper]{article}
 
\usepackage{color}
\usepackage{url}
\usepackage[T2A]{fontenc} 
\usepackage[utf8]{inputenc}
\usepackage{graphicx}
 
\usepackage[english,serbian]{babel}
\usepackage[unicode]{hyperref}
\hypersetup{colorlinks,citecolor=red,filecolor=green,linkcolor=blue,urlcolor=blue}
 
\title{Slučajevi upotrebe Udruženja u okviru zaposlenih}
 
 
\begin{document}
 
\maketitle
 
\section{Slučajevi upotrebe}
 
\subsection{Udruženja}
\subsubsection{Slučaj upotrebe: Zahtev za održavanje događaja}
\begin{enumerate}
    \item \textbf{Kratak opis:} Zaposleni koji ima ulogu administratora udruženja šalje zahtev za održavanje događaja.
    \item \textbf{Učesnici:}
        \begin{itemize}
            \item Zaposleni
        \end{itemize}
    \item \textbf{Preduslovi:} Sistem je u funkciji. Zaposleni ima pristup internetu i sistemu.
    \item \textbf{Postuslovi:} Zahtev za održavanje događaja je podnet.
    \item \textbf{Osnovni tok:}
        \begin{enumerate}
            \item Zaposleni otvara stranicu za udruženja.
            \item Sistem prikazuje listu svih postojećih udruženja.
            \item Zaposleni otvara stranicu udruženja čiji je on administrator.
            \item Zaposleni otvara stranicu za podnošenje zahteva za održavanje događaja.
            \item Zaposleni popunjava formular za zahtev.
            \item Zaposleni potvrđuje podnošenje zahteva.
            \item Sistem beleži zahtev za održavanje događaja.
        \end{enumerate}
    \item \textbf{Alternativni tokovi:}
        \begin{enumerate}
            \item \textbf{Formular nije ispravno popunjen.} Sistem obaveštava zaposlenog o greškama na formularu i vraća ga na korak (e).
        \end{enumerate}
    \item \textbf{Podtokovi:} /
    \item \textbf{Specijalni zahtevi:} /
    \item \textbf{Dodatne informacije:} /
\end{enumerate}
\end{document}
has context menu
Compose