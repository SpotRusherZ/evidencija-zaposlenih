\documentclass[a4paper]{article}

\usepackage{color}
\usepackage{url}
\usepackage[T2A]{fontenc} 
\usepackage[utf8]{inputenc}
\usepackage{graphicx}

\usepackage[english,serbian]{babel}
\usepackage[unicode]{hyperref}
\hypersetup{colorlinks,citecolor=red,filecolor=green,linkcolor=blue,urlcolor=blue}

\title{Slučajevi upotrebe Udruženja u okviru zaposlenih}


\begin{document}

\maketitle

\section{Slučajevi upotrebe}

\subsection{Udruženja}
\subsubsection{Slučaj upotrebe: Zahtev za osnivanje udruženja}
\begin{enumerate}
    \item \textbf{Kratak opis:} Zaposleni u kompaniji šalje zahtev za osnivanje udruženja.
    \item \textbf{Učesnici:}
        \begin{itemize}
            \item Zaposleni
        \end{itemize}
    \item \textbf{Preduslovi:} Sistem je u funkciji. Zaposleni ima pristup internetu i sistemu.
    \item \textbf{Postuslovi:} Zahtev za osnivanje udruženja je uspešno podnet i u statusu je čekanja za odobravanje. Administrator udruženja je obavešten.
    \item \textbf{Osnovni tok:}
        \begin{enumerate}
            \item Zaposleni otvara stranicu za udruženja.
            \item Klikom na dugme za osnivanje novog udruženja prikazuje se forma koja sadrži polje gde se bira tip udruženja.
            \item Za slučaj da je tip udruženja sportska sekcija, izvršava se podtok (a), za programersku sekciju podtok (b), a za ostalo podtok (c).
            \item Zaposleni popunjava formular.
            \item Zaposleni potvrđuje podnošenje zahteva.
            \item Sistem beleži zahtev za osnivanje udruženja i obaveštava administratora udruženja.
        \end{enumerate}
    \item \textbf{Podtokovi:}
        \item Sportska sekcija
            \begin{enumerate}
                \item Zaposlenom se prikazuje forma gde se bira sport i polje gde se navode dodatne informacije.
            \end{enumerate}
        \item Programerska sekcija
            \begin{enumerate}
                \item Zaposlenom se prikazuje forma gde se unosi programski jezik za koji želi da osnuje zajednicu i polje gde se navode dodatne informacije.
            \end{enumerate}
        \item Ostalo
            \begin{enumerate}
                \item Zaposlenom se prikazuje forma gde se navodi naziv udruženja koji želi da osnuje, polje gde se navodi opis udruženja i polje gde se navode dodatne informacije.
            \end{enumerate}
    \item \textbf{Alternativni tokovi:}
        \begin{enumerate}
            \item \textbf{Formular nije ispravno popunjen.} Sistem obaveštava zaposlenog o greškama na formularu i vraća ga na korak (e).
        \end{enumerate}
    \item \textbf{Podtokovi:} /
    \item \textbf{Specijalni zahtevi:} /
    \item \textbf{Dodatne informacije:} /
\end{enumerate}
\end{document}